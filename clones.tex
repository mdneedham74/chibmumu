\section{Clone requirments}
\label{sec:clones}
%
Duplicate tracks created by the reconstruction are inevitable at some
level. Great care is taken during the pattern recognition step in
\textbf{Brunel} to remove such clones. Such clones can be dangerous as
they can generate artificial mass peaks. To ensure that no clones
remain two requirements known from previous
studies to reject clones with minimal signal loss are applied \footnote{See for
example $https://twiki.cern.ch/twiki/bin/view/LHCbPhysics/CloneCuts$}. The code used for this is given below: 
%
\begin{scriptsize}
\begin{lstlisting}{}
int type1Clone(float tx1, float tx2, float ty1, float ty2) {
  return ((TMath::Abs(tx1 -tx2) < 0.0004) && (TMath::Abs(ty1 -ty2) < 0.0002));
}

int type2Clone(float tx1, float tx2, float ty1, float ty2, float qDivp1, float qDivp2  ){
  return ((TMath::Abs(tx1 -tx2) < 0.005)
  && (TMath::Abs(ty1 -ty2) <
  0.005) && TMath::Abs(qDivp1 - qDivp2) < 1e-6);
}

int t1Clone(float tx1, float tx2, float tx3, float tx4, 
                 float ty1, float ty2,float ty3, float ty4 ){

  
  if (type1Clone(tx1,tx2, ty1,ty2) == 1) return 1;
  if (type1Clone(tx1,tx3, ty1,ty3) == 1) return 1;   
  if (type1Clone(tx1,tx4, ty1,ty4) == 1) return 1;   
  if (type1Clone(tx2,tx3, ty2,ty3) == 1) return 1;  
  if (type1Clone(tx2,tx4, ty2,ty4) == 1) return 1;  
  if (type1Clone(tx3,tx4, ty3,ty4) == 1) return 1;  
  return 0;
}

int t2Clone(float tx1, float tx2, float tx3, float tx4, float ty1, 
             float ty2,float ty3, float ty4 , float  qp1, 
            float qp2, float qp3, float qp4){

  if (type2Clone(tx1,tx2, ty1,ty2, qp1, qp2) == true) return 1;
  if (type2Clone(tx1,tx3, ty1,ty3, qp1, qp3) == true) return 1;   
  if (type2Clone(tx1,tx4, ty1,ty4, qp1, qp4) == true) return 1;   
  if (type2Clone(tx2,tx3, ty2,ty3, qp2, qp3) == true) return 1;  
  if (type2Clone(tx2,tx4, ty2,ty4, qp2, qp4) == true) return 1;  
  if (type2Clone(tx3,tx4, ty3,ty4,qp3, qp4) == true) return 1;  
  return 0;
}
\end{lstlisting}{}
\end{scriptsize}

\clearpage
