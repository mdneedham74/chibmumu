\section{Jargon to be avoided}
\label{sec:jargon}
We try to avoid HEP-specific jargon, even if it is widely used in the literature. Many of the words listed below are imprecise and are used for exactly this reason. They do not require to define precisely what is being meant. If any of the words below appears in your paper,
think twice.
\begin{description}
\item[Cut] is HEP jargon. We prefer requirement. 
\item[Error.] ISO defines the error as the difference between the measured and true value, and the uncertainty as the estimate of this difference. We usually deal with {\it uncertainties}. {\it Error bar} is acceptable.
\item[Event.] Event is often used sloppily to mean the $pp$ collision, the whole bunch crossing, or the \B decay of interest. In most LHCb papers we use ``Event'' to mean the whole collision. The trigger selects events. Then we build {\it candidates}. The fit returns a signal yield (not a number of candidates). {\it Decay} can be used to mean signal.
\item[Extract.] Gold is extracted from mines. We determine signal yields, branching fractions\dots
\item[Final state.] Make sure your final states are really final, meaning they are what is being detected. So avoid \jpsi{}\Kp final states, for instance.
\item[\boldmath\invfb of data.] \invfb is not a unit of data. Say ``data corresponding to an integrated luminosity of 3\invfb''.
\item[Monte-Carlo] is a simulation technique. We refer to the simulated data and to the programme generating it as the {\it simulation}.  
\item[Mother, daughter, bachelor\etc] We try to stick to a gender-neutral language.  Mother is rarely needed. Daughter can be a decay product or a final state. Bachelor can be replaced by companion.
\item[Reweight.] {\it Weight}, unless the sample was already weighted.
\item[Statistics] to mean the size of the data sample. Replace with data sample size.
\item[Systematic] is an adjective. ``A systematic'' is thus incorrect and to be replaced by systematic uncertainty.
\end{description}
