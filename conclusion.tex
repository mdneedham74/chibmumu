\section{Conclusion}
\label{sec:conclusion}
%
In this note an emulator to estimate the mass resolution for the decay
$\chicone \rightarrow J/\psi \mu^+ \mu^-$ has been developed. The
emulator reproduces the resolution in the full simulation
 for this mode. Dividing data into regions of phase space
the agreement with the full simulation is generally found at the level
of a $5 \%$ or better . The worse level of agreement is at the $10 \%$ level
at high candidate $\pt$.  Tte model is flexible enough to
be used for similar topology modes where it reproduces the overall
resolution at the level of $3 \%$ or better. It is intended to use
this mode for studies of $\chi_b$ Dalitz decays. Though the work
desecribed here is adequate for that use-case various improvements
could be envisaged. First, more care could be taken in data-sample
selection to give a more uniform coverage of the phase space. Second,
the hyperparameters of the BDT have not been optimized beyond the
defaults suggested in TMVA. Finally, it would be better to train one
BDT rather than three which also takes into account the correlations
between the parameters. That would be possible in a framework such as
Tensorflow.

The regression model for the momentum resolution has wider
applicability and could be used directly in \textbf{RapidSim} or other
fast simulation packages. An insight gained from this study is the
importance of the resolution on the track slopes for modes with soft
tracks. This is greatly improved by application of a pointing constraint
to the PV in the decay tree fit.  This needs to considered when using
a fast simulation. To properly account for this requires either to
explicitly apply this constraint or to properly parameterize the
effect for different topologies.



\clearpage