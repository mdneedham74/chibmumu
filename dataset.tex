<<<<<<< HEAD
\section{Dataset and selection}
\label{sec:dataset}
 The simulation samples used in this study are summarized in Table
 \ref{tab:mc}. The samples were processed with \textbf{DaVinci}
 v44r6. The sample of the $\chicone
 \rightarrow J/\psi \mu^+ \mu^-$ is the main one used for this
 study. To parameterize  the momentum resolution in a more general way
 the $\Upsilon  \rightarrow \mu^+ \mu^-$ decay, with high momentum
 tracks is used. The other samples are used to validate the method and
 shows it more general applicability. 

\begin{table}[htb!]
\caption{\small Simulated data samples used in this study. }
\small
\begin{center}
\scalebox{0.8}{%
\begin{tabular}{l|c|c|c|c|c|c}
Mode & Event type & Year & Polarity &  DDDB & SIMCOND & Events  \\
\hline
$\chicone \rightarrow J/\psi \mu^+ \mu^-$ & 28144041 & 2016 & Up &
                                                                   dddb-20150724
                                            & sim-20161124-2-vc-mu100
                                                      & 2002668 \\
$\chicone \rightarrow J/\psi \mu^+ \mu^-$ & 28144041 & 2016 & Down &
                                                                   dddb-20150724
                                            & sim-20161124-2-vc-md100
                                                      &  2005998 \\
$\chictwo \rightarrow J/\psi \mu^+ \mu^-$ & 28144051 & 2016 & Up &
                                                                   dddb-20150724
                                            & sim-20161124-2-vc-mu100
                                                      &  2002004 \\
$\chictwo \rightarrow J/\psi \mu^+ \mu^-$ & 28144051 & 2016 & Down &
                                                                   dddb-20150724
                                            & sim-20161124-2-vc-md100
                                                      &  2000162 \\
$\Upsilon  \rightarrow \mu^+ \mu^-$ & 18112001 & 2016 & Up & dddb-20150724
                                                                   
                                            & sim-20161124-2-vc-mu100
                                                      &  539200 \\
$\Upsilon \rightarrow \mu^+ \mu^-$ & 18112001 & 2016 & Down & dddb-20150724
                                                                   
                                            &  sim-20161124-2-vc-md100
                                                      & 505564 \\

$\psi(2S) \rightarrow J/\psi \pi^+ \pi^-$  &  28144002 & 2011 & Up &
                                                                   
                                  dddb-20170721-1          & sim-20160614-1-vc-md100
                                                      & 731657 \\
$\psi(2S) \rightarrow J/\psi \pi^+ \pi^-$  &  28144002  & 2011 & Down
                                    &  dddb-20170721-1 
                                                                   
                                            & sim-20160614-1-vc-md100
                                                      & 729988  \\
$\psi(2S) \rightarrow J/\psi \pi^+ \pi^-$  &  28144002 & 2012 & Up &
                                                                   
                     'dddb-20170721-2                       & sim-20160321-2-vc-mu100
                                                      & 1568952 \\
$\psi(2S) \rightarrow J/\psi \pi^+ \pi^-$  & 28144002 & 2011 & Down &
                                                                   
                    'dddb-20170721-2                        & sim-20160321-2-vc-md100
                                                      &  1573450 \\


$X(3872) \rightarrow J/\psi \pi^+ \pi^-$  & 28144011 & 2016 & Up &
                                                                   
                                 dddb-20170721-3           & sim-20170721-2-vc-mu100
                                                      &  2010735\\
$X(3872) \rightarrow J/\psi \pi^+ \pi^-$  & 28144011 & 2016 & Down &
                                       dddb-20170721-3                            
                                            & sim-20170721-2-vc-md100
                                                      &  2000678 \\


\end{tabular}}
\end{center}
\label{tab:mc}
\end{table}

For these studies, the selection requirements are relatively
unimportant: all that is needed is a sample of good tracks covering
the entire spectrometer acceptance.  Loose selections are applied to all samples and all candidates are
required to be truth matched using the background category tool
\cite{Gligorov:1035682}. The selection criteria for $\chicone \rightarrow
J/\psi \mu^+ \mu^-$ are described in
Ref. ~\cite{Anderlini:2270922}. An important change made was to remove the
particle identification requirements.  This is important since
\textbf{isMuon} requires $p > 3 \gevc$ which is higher than the
acceptance of the spectrometer ($\sim 1.5 \gevc$). In addition, the inner
acceptance of the muon system does not match exactly that of the
tracker. Without these changes the emulator does not cover the full
acceptance and will consequently not be able to perform optimally for
modes involving particles other than muons.  Similar selections are applied for the $\psi(2S)$ and
$X(3872)$ modes. The $\Upsilon \rightarrow \mu^+ \mu^-$ candidates are
output by the \textbf{FullDSTDiMuonDiMuonHighMassLine} line.
=======
\section{Dataset}
\label{sec:dataset}
%
The full dataset collected by LHCb between 2011 and 2018 is used for
this analysis. This corresponds to 1~\invfb collected at 7\tev,
2~\invfb at 8\tev and 6 \invfb collected at 13\tev. For processing
\textbf{DaVinci v44r6p1} was used. The momentum scale calibration available in the latest database
tags is applied \footnote{The tool \textbf{TrackScaleState}.} to the data. This
ensures the momentum scale is correct to a precision of $3
\times 10^{-4}$.

The present of a $\Upsilon(nS) n = 1,2,3$ meson decaying to a dimuon
provides a distinctive signature for triggering with high efficiency. The data collected
from 2011-2016 went to the full DST. Events collected during this
period are requested to pass the following three trigger levels:
\begin{description}
\item[L0Muon] Events should pass either the $L0$ single or dimuon
  lines. The single muon line requires a single $\pt$ muon and cuts on
  the total number of hits in the SPD. The dimuon line requires the presence of two muon candidates
  with a product of $\pt$'s larger than a value that depends on the
  running period (typically $1.68 (\gevc)^2$) and less than 900 hits
  in the SPD.
\item[Hlt1DiMuonHighMass] This requires the presence of two well
  reconstructed tracks with hits in the muon system, momentum in
  excess of $6 \gevc$ and have $\pt > 0.5 \gevc$. The two muons must
  form a common vertex and have an invariant mass larger than $2.7
  \gevcc$. 
\item[HLT2DiMuonB] This confirms the \textbf{HLT1} decision with higher precision and requires
the invariant mass of the dimuon pair to be larger than  $4.7
  \gevcc$. 
\end{description}
These events are then stripped using the
\textbf{FullDSTDiMuonDiMuonHighMassLine} line. This tightens the
requirements on track and vertex quality applied in the trigger, in
particular requiring $p > 8 \gevc$ and applies an invariant mass cut
at $> 8.5 \gevc$. The stripping versions used are summarized in Table \ref{tab:strip}. 

\begin{table}[htb!]
\caption{\small Stripping datasets }
\begin{center}
\begin{tabular}{l|c}
Year & Stripping version  \\
\hline
2011 & 21r1 \\
2012 & 21 \\
2015 & 24r1 \\
2016 & 28r1 \\
\end{tabular}
\end{center}
\label{tab:strip}
\end{table}

From 2017 onwards the $\Upsilon(nS) n = 1,2,3$ lines moved to
\textbf{TURBO}. The  \textbf{L0} and \textbf{HLT1} lines remain the
same. At \textbf{HLT2}, the line
\textbf{Hlt2DiMuonUpsilonTurboDecision} is required and no stripping
line is necessary. The \textbf{Hlt2DiMuonUpsilonTurbo} line confirms
the \textbf{HLT1} decision and requires a dimuon pair with an invariant mass larger than $7.9
  \gevcc$. \textbf{TURBO4a} is used for 2017 and \textbf{TURBO5} for 2018.
>>>>>>> b5746ff0abefc10c12f8e4f931af1da1f187a7d6
