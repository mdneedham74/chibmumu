\section{Dataset}
\label{sec:dataset}
%
The full dataset collected by LHCb between 2011 and 2018 is used for
this analysis. This corresponds to 1~\invfb collected at 7\tev,
2~\invfb at 8\tev and 6 \invfb collected at 13\tev. For processing
\textbf{DaVinci v44r6p1} was used. The momentum scale calibration available in the latest database
tags is applied \footnote{The tool \textbf{TrackScaleState}.} to the data. This
ensures the momentum scale is correct to a precision of $3
\times 10^{-4}$.

The present of a $\Upsilon(nS) n = 1,2,3$ meson decaying to a dimuon
provides a distinctive signature for triggering with high efficiency. The data collected
from 2011-2016 went to the full DST. Events collected during this
period are requested to pass the following three trigger levels:
\begin{description}
\item[L0Muon] Events should pass either the $L0$ single or dimuon
  lines. The single muon line requires a single $\pt$ muon and cuts on
  the total number of hits in the SPD. The dimuon line requires the presence of two muon candidates
  with a product of $\pt$'s larger than a value that depends on the
  running period (typically $1.68 (\gevc)^2$) and less than 900 hits
  in the SPD.
\item[Hlt1DiMuonHighMass] This requires the presence of two well
  reconstructed tracks with hits in the muon system, momentum in
  excess of $6 \gevc$ and have $\pt > 0.5 \gevc$. The two muons must
  form a common vertex and have an invariant mass larger than $2.7
  \gevcc$. 
\item[HLT2DiMuonB] This confirms the \textbf{HLT1} decision with higher precision and requires
the invariant mass of the dimuon pair to be larger than  $4.7
  \gevcc$. 
\end{description}
These events are then stripped using the
\textbf{FullDSTDiMuonDiMuonHighMassLine} line. This tightens the
requirements on track and vertex quality applied in the trigger, in
particular requiring $p > 8 \gevc$ and applies an invariant mass cut
at $> 8.5 \gevc$. The stripping versions used are summarized in Table \ref{tab:strip}. 

\begin{table}[htb!]
\caption{\small Stripping datasets }
\begin{center}
\begin{tabular}{l|c}
Year & Stripping version  \\
\hline
2011 & 21r1 \\
2012 & 21 \\
2015 & 24r1 \\
2016 & 28r1 \\
\end{tabular}
\end{center}
\label{tab:strip}
\end{table}

From 2017 onwards the $\Upsilon(nS) n = 1,2,3$ lines moved to
\textbf{TURBO}. The  \textbf{L0} and \textbf{HLT1} lines remain the
same. At \textbf{HLT2}, the line
\textbf{Hlt2DiMuonUpsilonTurboDecision} is required and no stripping
line is necessary. The \textbf{Hlt2DiMuonUpsilonTurbo} line confirms
the \textbf{HLT1} decision and requires a dimuon pair with an invariant mass larger than $7.9
  \gevcc$. \textbf{TURBO4a} is used for 2017 and \textbf{TURBO5} for 2018.
