\section{Selection}
\label{sec:selection}
 %
No blind analysis was performed, rather the selection aims to be as similar as possible to the analysis of the $\chic
\rightarrow J/\psi \mu^+ \mu^-$ decay mode in Ref. \cite{Anderlini:2270922}. The
emphasis in the selection is on on \textit{track quality}.  

Further loose cuts are applied to the dimuon pair created by the stripping
or TURBO line to increase the purity. To ensure good track quality the
cut in the ghost probability is tightened to 0.1, the $\pt$ of both
muons is required to be larger than $1 \gevc$ and $p < 500 \gevc2$.  Both tracks are required to be
well identified muons by requiring \textbf{ProbNNmu}$>0.2$ for Run 1
data and \textbf{ProbNNmu}$>0.3$ in Run2. In the study of $\chic
\rightarrow J/\psi \mu^+ \mu^-$ decays \cite{Anderlini:2270922} it was found that background
was significantly reduced by requiring $\eta < 4.9$ and consequently this
requirement is applied here.  Finally, both tracks should
be within the acceptance of the spectrometer ($|tx| < 0.3
\textrm{mrad}$ and $|ty| < 0.25 \textrm{mrad}$). A mass window
corresponding to a roughly $\pm 3 \sigma$ is applied to the $\Upsilon$
candidate. The size of the window used for each $\Upsilon$ state is
summarized in Table \ref{tab:umasscut}. 
\begin{table}[htb!]
\caption{\small $\Upsilon$ mass window }
\begin{center}
\begin{tabular}{l|c}
State & Window [$\mevcc$]  \\
\hline
$\Upsilon(1S)$ & 125 \\
$\Upsilon(2S)$ & 130 \\
$\Upsilon(3S)$ & 140 \\
\end{tabular}
\end{center}
\label{tab:umasscut}
\end{table}

To select muons from the virtual photon in $\chi_b \rightarrow
\Upsilon \mu^+ \mu^-$ the same cuts are applied as for the muons from
the $\Upsilon$, apart from the $\pt$ and $p$ requirements which are
loosened to $200 \mevc$ and $1.5 \gevc$ respectively. Soft pions are
selected using the same kinematic, geometric and track quality
criteria and by requiring  \textbf{ProbNNpi}$>0.2$ and vetoing
possible decays in flight using \textbf{isMuon} as a veto. In addition
to the standard clone removal applied in the reconstructed further
requirements are made as described in Appendix~\ref{app:clones}. 
 
Pairs of muons (pions) with opposite charge are combined with
$\Upsilon(nS)$ candidates to form $\chi_b$ ($\Upsilon(nS+1)$)
candidates. A kinematic fit is applied to each candidate with the
$\Upsilon$ mass constrained to the nominal value and a constraint on
the pointing of the candidate to the primary vertex. The $\chi^2/NDOF$
of this fit is required to be less than four. The estimate of the mass
uncertainty on the candidate given by the track fit is required to be
less than $5 \mevc$. To define a fiducial volume where the simulation
is reliable the $\pt$ of the candidate is required to be less than $60
\gevc$ and the rapidity restricted to the range 2.2 - 4.5. 