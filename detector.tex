\section{Detector and simulation}
\label{sec:Detector}
%
The \lhcb detector~\cite{Alves:2008zz,LHCb-DP-2014-002} is a
single-arm forward spectrometer covering the \mbox{pseudorapidity}
range $2<\eta <5$, designed for the study of particles containing
\bquark or \cquark quarks. The detector includes a high-precision
tracking system consisting of a silicon-strip vertex detector surrounding the $pp$ interaction
region~\cite{LHCb-DP-2014-001}, a large-area silicon-strip
detector (TT) located upstream of a dipole magnet with a bending power of
about $4{\mathrm{\,Tm}}$, and three stations of silicon-strip
detectors and straw drift tubes~\cite{LHCb-DP-2013-003} placed
downstream of the magnet. The tracking system provides a measurement
of momentum, \ptot, of charged particles with a relative uncertainty
that varies from 0.5\% at low momentum to 1.0\% at 200\gevc. As
described in Refs.~\cite{LHCb-PAPER-2012-048,LHCb-PAPER-2013-011} the
momentum scale is calibrated using samples of $\jpsi \rightarrow \mup
\mun$ and $\Bu \rightarrow \jpsi \Kp$ decays collected concurrently
with the data sample used for this analysis. The accuracy of this
procedure is estimated to be $3 \times 10^{-4}$ using samples of other
fully reconstructed $\bquark$-hadrons, $\Upsilon$, $\psitwos$ and $\KS$ decays. The minimum distance of a track to a primary vertex (PV), the impact
parameter (IP), is measured with a resolution of $(15+29/\pt)\mum$,
where \pt is the component of the momentum transverse to the beam,
in\,\gevc. 

Different types of charged
hadrons are distinguished using information from two ring-imaging Cherenkov (RICH)
detectors. Photons, electrons and hadrons are
identified by a calorimeter system consisting of scintillating-pad and
preshower detectors, an electromagnetic calorimeter and a hadronic
calorimeter. Muons are  identified by a system composed of alternating
layers of iron and multiwire proportional chambers~\cite{LHCb-DP-2012-002}.  

The online event selection is performed by a
trigger~\cite{LHCb-DP-2012-004}, which consists of a hardware stage,
based on information from the calorimeter and muon 
systems, followed by a software stage, where  a full event
reconstruction is made. Candidate events are required to pass the
hardware trigger, which selects muon and dimuon candidates with high
$\pt$ based upon muon system information. The subsequent software trigger is composed of two stages. The first performs a
partial event reconstruction and requires events to have two well-identified oppositely charged muons with an invariant mass larger
than $2.7 \gevcc$. The second stage performs a full event
reconstruction.  Events are retained for further processing if they
contain a displaced $\jpsi \rightarrow \mu^+ \mu^-$ candidate. The
decay vertex is required to be well separated from each reconstructed
PV of the proton-proton interaction by requiring the distance between
the PV and the $\jpsi$ decay vertex  divided by its uncertainty to be greater than three.

To study the properties of the signal and the most important backgrounds, $pp$ collisions are generated using 
\pythia~\cite{Sjostrand:2006za,*Sjostrand:2007gs}  with a specific
\lhcb configuration~\cite{LHCb-PROC-2010-056}.  Decays of hadronic
particles are described by \evtgen~\cite{Lange:2001uf}, in which
final-state radiation is generated using
\photos~\cite{Golonka:2005pn}. The 
interaction of the generated particles with the detector, and its
response, are implemented using the \geant
toolkit~\cite{Allison:2006ve, *Agostinelli:2002hh} as described in
Ref.~\cite{LHCb-PROC-2011-006}. For the study of the line shape it is
important that the simulation models well the mass resolution. For selected samples of $\Bu \rightarrow
\jpsi K^+$, $\Bz \rightarrow \jpsi \Kp \pim$, $\Bsb \rightarrow
\jpsi \phi$ and $\Bu \rightarrow \jpsi  \Kp \pip \pim$ candidates
the simulation reproduces the observed mass resolution at the
level of $5 \%$. To further improve the agreement between data and simulation
resolution scale factors are determined using a large sample of
$\psitwos \rightarrow \jpsi
\pip \pim $ decays collected concurrently with the
$\chicone(3872)$ sample.

